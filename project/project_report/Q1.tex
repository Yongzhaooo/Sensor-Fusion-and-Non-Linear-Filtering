\section{Q1: Choices of input and output}
\subsection{As input}

\textbf{Pros of selecting $u_k = w_k$ (using angular velocities as inputs):}

1. Accuracy: Gyroscope measurements of angular velocities are typically more accurate and less prone to noise compared to other sensor measurements. Using them as inputs can improve the accuracy of the estimation process.

2. Reduced computational complexity: Using angular velocities as inputs reduces the dimensionality of the state vector, simplifying the estimation algorithm and reducing computational requirements. This leads to faster execution and lower resource consumption.

\textbf{Cons of selecting $u_k = w_k$:}

1. Susceptibility to gyro drift: Gyroscopes can experience drift over time, leading to errors in the estimated orientation. Extended operation without external reference or correction can cause gradual deviation from the true orientation.

2. Lack of absolute reference: Using angular velocities as inputs does not provide an absolute reference for the system's orientation. The estimated orientation may drift over time, especially without external measurements or references to correct cumulative errors.

In fact, in subsequent experiments, it was observed that the gyroscope exhibits high accuracy and is subjected to minimal disturbances, making it well-suited for use as an input.

\subsection{As state}

Including angular velocities in the state vector is beneficial in the following situations:

1. Dynamic Systems: If the system being modeled has complex dynamics that cannot be accurately captured solely by measurements, including angular velocities as state variables provides a more comprehensive representation of the system's behavior.

2. Noisy Measurements: If the measurements of angular velocities are noisy or subject to significant disturbances, including them as state variables helps mitigate the impact of measurement uncertainties.

3. Biased Measurements: When measurements of angular velocities suffer from biases, incorporating them as state variables and estimating the biases allows the estimation algorithm to compensate for these biases and provide more accurate results.