\documentclass[12pt]{article} 
\usepackage{geometry} 
\geometry{a4paper} 
\linespread{1.1} % Line spacing


% FIGURES AND FLOATS
\usepackage{graphicx} % Required for including pictures
\usepackage{float} % Allows putting an [H] in \begin{figure} to specify the exact location of the figure
\usepackage{wrapfig} % Allows in-line images such as the example fish picture
\usepackage[font={small,it}]{caption}
\usepackage{subcaption}
\usepackage{epstopdf}
\graphicspath{{images/}}
\usepackage{diagbox}

% MATH
\usepackage{amssymb}
\usepackage{amsmath}
\usepackage{algorithm}
\usepackage[noend]{algpseudocode}

% OTHER
\usepackage[]{mcode}
\usepackage{enumerate}
\usepackage{xcolor}


%character
\usepackage{indentfirst} % 强制要求第一个段落也进行缩进
%段间距
\setlength{\parskip}{1ex plus 0.5ex minus 0.5ex}
%%%%%%%%%%%%%%%%%%%%%%%%%%%%%%%%%%%%%%%%%%%%%%%%%%%%%%%%%%%%
\usepackage{listings} 

%段首缩进
\setlength{\parindent}{2em} 

\definecolor{DeepPink}{RGB}{255, 20, 147}
\definecolor{LightSlateBlue}{RGB}{132, 112, 255}
\definecolor{ForestGreen}{RGB}{34, 139, 34}

\lstset{
  language=Matlab,  %代码语言使用的是matlab
  frame=shadowbox, %把代码用带有阴影的框圈起来
  rulesepcolor=\color{red!20!green!20!blue!20},%代码块边框为淡青色
  keywordstyle=\color{blue!90}\bfseries, %代码关键字的颜色为蓝色,粗体
  commentstyle=\color{ForestGreen}\textit,    % 设置代码注释的颜色
  showstringspaces=false,%不显示代码字符串中间的空格标记
  numbers=left, % 显示行号
  numberstyle=\tiny,    % 行号字体
  stringstyle=\ttfamily, % 代码字符串的特殊格式
  basicstyle={\small\ttfamily},
  breaklines=true, %对过长的代码自动换行
  extendedchars=false,  %解决代码跨页时,章节标题,页眉等汉字不显示的问题
  texcl=true,
  % basicstyle=\fontfamily{Microsoft YaHei}\selectfont\footnotesize, % 设置字体族为微软雅黑,字号为footnotesize
}

\lstset{breaklines}%自动将长的代码行换行排版

\lstset{extendedchars=false}%解决代码跨页时,章节标题,页眉等汉字不显示的问题
%%%%%%%%%%%%%%%%%%%%%%%%%%%%%%%%%%%%%%%%%%%%%%%%%%%%%%%%%%%%
\title{\LARGE Solution to analysis in Home Assignment 4 \\  \vspace{1cm}\Large }
\author{Yongzhao Chen(yongzhao@chalmers.se)}
\date{\vspace{8cm}\today}
\begin{document}
\maketitle
\thispagestyle{empty}
\newpage
%%%%%%%%%%%%%%%%%%%%%%%%%%%%%%%%%%%%%%%%%%%%%%%%%%%%%%%%%%%%
\section{Smoothing}
\subsection{Task a}

In this part, mainly use the code from HA3 Q3, but use \texttt{nonLinRTSsmoother} function to generate the filtered output and the smoothed output.

And I realized my last HA3 had some bug, after the fix, the tuning value is chosen as :

\begin{equation}
    \begin{aligned}
        sigma_v = 10;\\
sigma_w = pi/180*6;\nonumber
    \end{aligned}
\end{equation}

The result shown in figure \ref{taska}, legends are detailed. To be more readable and to concentrate on the differences, the picture is zoomed in:

\begin{figure}[H]
 \centering
 \includegraphics[width=0.7\textwidth]{images/taska.png}
 \caption{Task a: Comparison}
 \label{taska}
\end{figure}

\subsubsection{Conclusion}

From figure \ref{taska}, the green curve, which represents the smoothed output is closer to the true position line compare with the red curve, which represents the filtered output.

And from the $ 5_{th} $ curves, the covariances of the smoothed output are smaller than the filtered one.

\subsection{Task b}

I increased the value at $k=300$  by $ 10\% $ when generating true state $ X $ to see the result:

\begin{lstlisting}
    for i = 2:K+1
    if i ==300
    X(:,i) = 1.1*coordinatedTurnMotion(X(:,i-1),T);
    X(5,i) = 1.1*omega(i);
    else
    X(:,i) = coordinatedTurnMotion(X(:,i-1),T);
    X(5,i) = omega(i);
    end
end
\end{lstlisting}

\begin{figure}[H]
 \centering
 \includegraphics[width=0.7\textwidth]{images/taskb.png}
 \caption{Task b: Manually outlier}
 \label{Taskb}
\end{figure}

From figure \ref{Taskb}, the filtered output of the red curve follows the change of my manual outlier value, and has a big covariance at that point, while the smoothed output of the green curve goes less to the wrong value and has a smaller covariance compare to the filtered curve.

In summary, the filter is trying to incorporate the most recent measurement, including the outlier, into its estimate, while the smoother takes into account both past and future measurements when updating its estimate. As a result, the smoother can mitigate the effect of the outlier by considering the overall trajectory of the object and recognizing that the outlier is inconsistent with the other measurements.
\section{Non-linear Kalman filtering}

\subsection{a}

Part a and b have lots of pictures, I listed them and discuss together.

\begin{figure}[H]
 \centering
 \includegraphics[width=0.95\textwidth]{images/normala.png}
 \caption{Normal Case}
 \label{normal}
\end{figure}


\begin{figure}[H]
 \centering
 \includegraphics[width=0.95\textwidth]{images/case1.png}
 \caption{Extreme Situation (all bad)}
 \label{allbad}
\end{figure}

\begin{figure}[H]
 \centering
 \includegraphics[width=0.95\textwidth]{images/extremea.png}
 \caption{Extreme Situation (EKF bad)}
 \label{EKFbad}
\end{figure}

\subsection{b}

\begin{figure}[H]
 \centering
 \includegraphics[width=0.95\textwidth]{images/extreme.png}
 \caption{Extreme Case 2}
 \label{extremecase2}
\end{figure}

\begin{figure}[H]
 \centering
 \includegraphics[width=0.95\textwidth]{images/Normalcase3.png}
 \caption{Normal Case 3}
 \label{2b}
\end{figure}

\subsection{discuss a\&b}

For this question, I tried a very large number of times. In the vast majority of cases, there is no significant difference between EKF and UKF/CKF, it achieves tracking very well, and its covariance is not significantly larger compared to the other two, and only in a few cases can be distinguished by the naked eye, like in figure \ref{allbad}.

The covariance is well represented by the unvertainty, can be seen in Figure \ref{normal}.

All methods are poor when the two trajectories are nearly coincident, showed in \ref{allbad}

The failure of the EKF can be observed when the car surrounds the sensor at a relatively close distance. This is because at very close distances to the sensor, as if two points were taken on a circle of small radius to connect the lines to represent the arc between them, the nonlinear of the car is so severe that the EKF fails. These cases showed in \ref{EKFbad} and \ref{extremecase2}

To be honest, although the extreme situation will fail, in my experience, the reliability of the EKF is amazing.

\subsection{c}

Upon the figures, I print the mean and covariance of the estimation errors. In this way it can be much more clear to evaluate the quality of the filter.

\begin{figure}[H]
 \centering
 \includegraphics[width=0.95\textwidth]{images/hiscase1.png}
 \caption{Case 1}
 \label{c1}
\end{figure}

\begin{figure}[H]
 \centering
 \includegraphics[width=0.95\textwidth]{images/hiscase2.png}
 \caption{Case 2}
 \label{c2}
\end{figure}

\begin{figure}[H]
 \centering
 \includegraphics[width=0.95\textwidth]{images/hiscase3.png}
 \caption{Case 3}
 \label{c3}
\end{figure}

From figures above, concentrate on the Mean value ( the closer to zero, the better the quality), we can see that EKF has the worst performance within three methods, while the UKF and the CKF has nearly the same performance. 

The histogram of EKF has nothing bussiness with Gaussian distribution, and that explains why its mean so far away from zero. From the histograms of UKF and CKF, we can see the histograms for $ x $ look quite like Gaussian. But when turns to y, since it is a nonlinear measurement model, the more the nonlinear, the less the histograms look like Gaussian.

And the normalpdf curves are drawn within the  $ \pm 5 \sigma $, since from the figures in Task a we can see the measurement points are so scattered that the nosies are too big, which leads to many points do not lie within the $ \pm 5 \sigma $ zone. So even with normalization, the curve can only looks quite like Gaussian but not coincident to it like the previous homeworks.
\section{Design the EKF time update step}

\subsection{Task 3}

To derive a discretized model from the continuous time model in equation (5), we can solve the differential equation and use the relation $exp(A) \approx I + A$ to obtain the discretized form. Here's the derivation:

The continuous time model is :
\begin{equation}
    \begin{aligned}
        \dot q(t)=\frac12S\left(w_{k-1}+v_{k-1}\right)q(t)
    \end{aligned}
\end{equation}

Use the relation $exp(AΔt) \approx I + AΔt$ to solve ODE:

\begin{equation}
    \begin{aligned}
        q(t+T)=\exp\left(\frac{1}{2}S\left(w_{k-1}+v_{k-1}\right)T\right)\cdot q(t) = \left[\mathbf{I}+{\frac{T}{2}}S\left(w_{k-1}+v_{k-1}\right)\right]\cdot q(t)\\
    \end{aligned}
\end{equation}

$T $ is the sample time interval. And the form can be translated into a discretized style:

\begin{equation}
    \begin{aligned}
        q_k&=\mathbf{I}\cdot q_{k-1}+\frac{T}{2}S\left(w_{k-1}+v_{k-1}\right)\cdot q_{k-1}\\
        &=\left(\mathbf{I}+{\frac{T}{2}}S\left(w_{k-1}\right)\right)\cdot q_{k-1}+{\frac{T}{2}}S\left(v_{k-1}\right)\cdot q_{k-1}
    \end{aligned}
\end{equation}

So we get:
\begin{equation}
    \begin{aligned}
        F(\omega_{k-1}) = \mathbf{I}  + \frac{T}{2}\cdot S(\omega_{k-1})\\
G(\hat{q}_{k-1}) = \frac{T}{2} \cdot S(\hat{q}_{k-1})\\
    \end{aligned}
\end{equation}




\subsubsection{Reason for Discretize}

In the EKF, the prediction step involves propagating the state estimate and covariance from the previous time step to the current time step. This propagation is typically done using the continuous-time dynamic model, which can be linearized around the current state estimate. However, linearizing the model can introduce errors, especially for highly nonlinear systems.

To address this issue, the discretized model derived using the approximation techniques provides an alternative approach for the prediction step in the EKF. By discretizing the continuous-time dynamic model, we can directly apply it in the discrete-time domain without the need for linearization.

\subsection{Task 4}

If there are angular velocities, update the estimate and covariance as motion model, in function \texttt{tu\_qw}.

Once $ v_k $ is missing, use the same consideration as in homework 2, skip the update for the current time, and keep the state and covariance value as the latest update value.

\subsection{Task 5}

In the function \texttt{Task5\_filterTemple}, I utilized the \texttt{tu\_qw } and \texttt{mu\_normalizeQ} functions for the gyroscope sensor.

As observed and analyzed previously, the gyroscope provides accurate angular velocities but cannot obtain the absolute orientation.

To address this, I established the initial flat state by starting with the phone facing left and standing on its left edge. However, during the process, it consistently exhibits an offset compared to the orientation measurement, which is displayed as 'Google'.

Additionally, the gyroscope is prone to drifting due to this bias. To demonstrate this behavior, I conducted a procedure where I placed the phone on a table, shook it for a period of time, and then returned it to its original position. I repeated this procedure twice, and the resulting drift process is clearly depicted in Figure \ref{fig:drift-process}.

\begin{figure}[H]
    \centering
    \begin{subfigure}[H]{0.3\textwidth}
        \includegraphics[width=\textwidth]{images/beginning.png}
        \caption{Begin}
        \label{fig:begin}
    \end{subfigure}
    \hfill
    \begin{subfigure}[H]{0.3\textwidth}
        \includegraphics[width=\textwidth]{images/drift1.png}
        \caption{Firstshake}
        \label{fig:drift1}
    \end{subfigure}
    \hfill
    \begin{subfigure}[H]{0.3\textwidth}
        \includegraphics[width=\textwidth]{images/drift2.png}
        \caption{Secondshake}
        \label{fig:drift2}
    \end{subfigure}
    \caption{Drift Process}
    \label{fig:drift-process}
\end{figure}




\section{MMSE and MAP estimators}

\subsection{a}

\begin{figure}[H]
 \centering
 \includegraphics[width=0.7\textwidth]{images/graphfor4a.png}
 \caption{Sample 3000 times }
 \label{4a}
\end{figure}

Since $\theta$ is equally likely to be -1 or 1, the histogram of y should show two normal distributions with means -1 and 1 and equal variances $\sigma^2 = 0.25$. The overall shape of the histogram should be a bimodal distribution, with two peaks at -1 and 1 and equal weights. This bimodal distribution represents a mixture of two normal distributions, each with a different mean and the same variance.

\subsection{b}

\emph{Note: In my homework, solving b is after c. So I used the conclusions in c to prove b while.}

In this problem, we are given that $y = \theta + w$, where $w \sim N(0, 0.25)$ is a normally distributed noise term with mean 0 and variance 0.25. Therefore, the probability density function of y given theta is given by:

$$p(y | \theta) = \frac{1}{\sqrt{2\pi\sigma^2}} \exp \left(-\frac{(y - \theta)^2}{2\sigma^2}\right)$$

From c, we get:

$$p(y) = 0.5 \cdot \frac{1}{\sqrt{2 \pi \sigma^2}} \exp \left(-\frac{(y - 1)^2}{2 \sigma^2}\right) + 0.5 \cdot \frac{1}{\sqrt{2 \pi \sigma^2}} \exp \left(-\frac{(y + 1)^2}{2 \sigma^2}\right)$$

Substituting $y=0.7$ and $sigma^2$ = 0.25, we obtain:
\begin{lstlisting}
q = @(theta,y) 1/sqrt(2*pi*sigma2)*exp(-(y-theta)^2/(2*sigma2));
q(1,0.7)
ans =

    0.6664


f = @(y) 0.5*1/sqrt(2*pi*sigma2)*exp(-(y-1)^2/(2*sigma2))+0.5*1/sqrt(2*pi*sigma2)*exp(-(y+1)^2/(2*sigma2));
f(0.7)

ans =

    0.3345
\end{lstlisting}

so:

$p(\theta=1 | y=0.7) = \frac{p(y=0.7 | \theta=1) p(\theta=1)}{p(y=0.7)} = \frac{0.6664\times0.5}{0.3345}=0.9961$

$$p(\theta=-1 | y=0.7) = 0.0039$$

So I guess $ \theta $ =1.

\subsection{c}

To prove that $p(y)$ is a mixture of two normal distributions with means $-1$ and $1$ and the same variance $\sigma^2$, we can use the law of total probability.

In this problem, we can partition the sample space of $y$ into two mutually exclusive events: $y = \theta + w$ where $\theta = 1$ and $\theta = -1$, where $w \sim \mathcal{N}(0, \sigma^2)$ is a normally distributed noise term with mean $0$ and variance $\sigma^2$. 

Using the law of total probability, we can express $p(y)$ as a mixture of the two normal distributions as follows:

$$p(y) = p(y | \theta = 1) p(\theta = 1) + p(y | \theta = -1) p(\theta = -1)$$

where $p(y | \theta = 1)$ is the probability density function of the normal distribution with mean $1$ and variance $\sigma^2$, and $p(y | \theta = -1)$ is the probability density function of the normal distribution with mean $-1$ and variance $\sigma^2$.

Substituting the expressions for the two probability density functions, we obtain:

$$p(y) = 0.5 \cdot \frac{1}{\sqrt{2 \pi \sigma^2}} \exp \left(-\frac{(y - 1)^2}{2 \sigma^2}\right) + 0.5 \cdot \frac{1}{\sqrt{2 \pi \sigma^2}} \exp \left(-\frac{(y + 1)^2}{2 \sigma^2}\right)$$

Therefore, $p(y)$ is a mixture of two normal distributions with means $-1$ and $1$ and the same variance $\sigma^2$.

\subsection{d}

The Bayesian rule says:
\begin{equation}
    \begin{aligned}
        p(\theta|y)= \frac{p(y|\theta)}{p(y)}\nonumber
    \end{aligned}
\end{equation}

Already known $ p(y) $ in question c and $ p(y|\theta) $ in question b, so:

\begin{equation}
    \begin{aligned}
        p(\theta|y)&=\begin{cases}\frac{\frac{1}{2}\frac{1}{\sqrt{2\pi}\sigma}\exp\{-\frac{1}{2\sigma^2}(y-1)^2\}}{\frac{1}{2}\frac{1}{\sqrt{2\pi}\sigma}\exp\{-\frac{1}{2\sigma^2}(y-1)^2\}+\frac{1}{2}\frac{1}{\sqrt{2\pi}\sigma}\exp\{-\frac{1}{2\sigma^2}(y+1)^2\}}& \text{if } \theta =1\\ \frac{\frac{1}{2}\frac{1}{\sqrt{2\pi}\sigma}\exp\{-\frac{1}{2\sigma^2}(y+1)^2\}}{\frac{1}{\sqrt{2\pi}\sigma}\exp\{-\frac{1}{2\sigma^2}(y-1)^2\}+\frac{1}{2}\frac{1}{\sqrt{2\pi}\sigma}\exp\{-\frac{1}{2\sigma^2}(y+1)^2\}}&\text{if } \theta=-1\end{cases}\\
        &=\begin{cases}\frac{\exp\{\frac{y}{\sigma^2}\}}{\exp\{\frac{y}{\sigma^2}\}+\exp\{-\frac{y}{\sigma^2}\}}&\text{if } \theta =1\\ \frac{\exp\{\frac{-y}{\sigma^2}\}}{\exp\{\frac{y}{\sigma^2}\}+\exp\{-\frac{y}{\sigma^2}\}}&\text{if } \theta =-1\end{cases}\\
        &\text{where  }  \sigma^2=0.25\nonumber
    \end{aligned} 
\end{equation}

\subsection{e}

\begin{equation}
    \begin{aligned}
        \:\hat{\theta}_{M M S E}&=\sum_{\theta}\theta\operatorname*{Pr}\{\theta|y\}.\:\\
        &=p(\theta=1|y)-p(\theta=-1|y)\\
        &=\dfrac{\exp\frac{y}{\sigma^2}-\exp-\frac{y}{\sigma^2}}{\exp\frac{y}{\sigma^2}+\exp-\frac{y}{\sigma^2}}\\
        &=\:\:\frac{2\sinh\left(\frac{y}{\sigma^2}\right)}{2\cosh\left(\frac{y}{\sigma^2}\right)}=\tanh\left(\frac{y}{\sigma^2}\right)=tanh(4y)\nonumber
    \end{aligned}
\end{equation}

\subsection{f}

\begin{equation}
    \begin{aligned}
        \hat{\theta}_{MAP}&=\arg\max\limits_{\theta=\pm1}\pi_y(\theta)\nonumber\\
        &=\begin{cases}1& \text{if } \frac{\exp\frac{y}{\sigma^2}}{\exp\frac{y}{\sigma^2}+\exp-\frac{y}{\sigma^2}}\geq\frac{\exp-\frac{y}{\sigma^2}}{\exp\frac{y}{\sigma^2}+\exp-\frac{y}{\sigma^2}}\\-1& \text{if } \frac{{\exp}\frac{y}{\sigma^2}}{\exp\frac{y}{\sigma^2}+\exp-\frac{y}{\sigma^2}}<\frac{{\exp}-\frac{y}{\sigma^2}}{\exp\frac{y}{\sigma^2}+\exp-\frac{y}{\sigma^2}}\end{cases}\\
        &=\begin{cases}1& \text{if } y\geq 0\\-1 & \text{if } y \leq 0 \end{cases}
    \end{aligned}
\end{equation}

\subsection{g}
\subsubsection{1}

In 4b), we made the guess that $\theta$ is 1. Meanwhile, for the specific value of $y=0.7$, both the MMSE estimator and the MAP estimator would predict that $\theta$ is 1. In this case, our guess in 4b coincides with both the MMSE and MAP estimators.

\subsubsection{2}

In this problem, the MMSE and MAP estimators for $\theta$ are different. The MMSE estimator is given by $\hat{\theta}_{MMSE} = \tanh(4y)$, while the MAP estimator is given by $\hat{\theta}_{MAP} = \operatorname{sgn}(y)$.

The MMSE estimator minimizes the expected squared error between the estimated value of $\theta$ and the true value, while the MAP estimator maximizes the posterior probability of $\theta$ given the observation $y$. In this case, the MMSE estimator and the MAP estimator make different decisions when $y$ is close to 0.

If $y$ is positive, both the MMSE and MAP estimators predict that $\theta$ is 1. If $y$ is negative, the MMSE estimator predicts that $\theta$ is -1, while the MAP estimator predicts that $\theta$ is 1. This is because the MMSE estimator takes into account the entire posterior distribution of $\theta$, while the MAP estimator only considers the most probable value of $\theta$.


%%%%%%%%%%%%%%%%%%%%%%%%%%%%%%%%%%%%%%%%%%%%%%%%%%%%%%%%%%%%%
\end{document}
